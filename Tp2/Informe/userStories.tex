
\thispagestyle{empty}

\vspace{4cm plus 1cm minus 1cm}

\section{User Stories}

%CU 1.recibiendo INFORMACIÓN de sensores
%Requerimiento: REQ 1
%Actor: sensor
%Descripción: el sistema recibe datos de los sensores (por medio de las TRs) 
\linea \subsection*{Recibiendo información de sensores}
Un dato que se sensa en un sensor de una Terminal Remota debe llegar a los modelos que lo requieran y a la central provincial a la que pertenezca.
\subsubsection*{Caso de Aceptación}
Completar...

%CU 2.Enviando datos a ec
%Requerimiento: REQ 1
%Actor: Timer de TR
%Descripción: Cada un minuto cada TR manda los datos recopilados hacia la EC
\linea \subsection*{Enviando datos a ec}
Todo dato recopilado por una terminal remota debe llegar a la estacion provincial que pertenezca.
\subsubsection*{Caso de Aceptación}
Completar...

%CU 3.recibiendo y analizando sms
%Requerimiento: REQ 1
%Actor: Servidor GSM
%Descripción: La EC recibe un SMS (por medio del servidor GSM) y valida que este haya sido enviado por una TR.
\linea \subsection*{Recibiendo y analizando sms}
Todo mensaje recibido por una estacion provincial debe ser verificado.
\subsubsection*{Caso de Aceptación}
Completar...

%CU 4.almacenando INFORMACIÓN de sensores
%Requerimientos: REQ 1
%Actor: abstracto
%Descripción: Se almacenan en el sistema los datos que llegaron a la EC.
\linea \subsection*{Almacenando informacion de sensores}
Todo dato que arriba a una estacion provincial no puede ser perdido.
\subsubsection*{Caso de Aceptación}
Completar...

%CU 5.Actualizando agenda de sensores
%Requerimiento: REQ 2
%Actor: Estación Central
%Descripción: el sistema define los intervalos de tiempo que debe esperar un sensor para samplear los datos y enviarlos a la estación central.
\linea \subsection*{Actualizando agenda de sensores}
Una estacion central debe ser capaz de definir la frecuencia de los sensores que tiene asignados.
%%%%%%%
% Duda: muchos modelos usan los sensores, la frecuencia de los mismos se ve afectada por todos o por uno solo ?? 
% Y si es por todos: cómo se debe tratar las inconsistencias ??
%%%%%%%
\subsubsection*{Caso de Aceptación}
Completar...

%CU 6.administrando energia de trs
%Requerimiento: REQ 3
%Actor: componente externa
%Descripción: el sistema selecciona la fuente de energía según los datos de contexto que le provee la componente externa
\linea \subsection*{Administrando energía de TRs}
Las terminales remotas deben poder administrar su propia energía automáticamente.
\subsubsection*{Caso de Aceptación}
Completar...

%CU 7.Modificando Reglas del modelo matemático
%Requerimiento: REQ 6
%Actor: Operario
%Descripción: el operario modifica o ingresa nuevas reglas al sistema y este las integra al modelo de datos.
\linea \subsection*{Modificando Reglas del modelo matemático}
Como JONATHAN quiero poder modificar las reglas del modelo matemático.
\subsubsection*{Caso de Aceptación}
Completar...

%CU 8.aplicando Reglas de modelo matemático
%Requerimiento: REQ 6
%Actor: Estación Central
%Descripción: el sistema se encarga de aplicar las reglas definidas para obtener información predictiva a partir de los datos almacenados.
\linea \subsection*{Aplicando reglas del modelo matemático}
Las reglas de los modelos matemáticos se deben poder aplicar con distintos métodos.
\subsubsection*{Caso de Aceptación}
Completar...

%CU 9.detectando (posible) caída de tr
%Requerimiento: REQ 4
%Actor: Estación Central
%Descripción: el sistema se encargará de detectar, cuando ocurra, si una TR está caída.
\linea \subsection*{Detectando (posible) caída de terminal remota}
El sistema debe encargarse de detectar automáticamente cuando una termianl se cae.
\subsubsection*{Caso de Aceptación}
%% El sistema ¿quien? ¿La estacion prov que sea?
Completar...

%CU 10.subsanando caída de tr
%Requerimiento: REQ 4
%Actor: abstracto
%Descripción: una vez detectada la caída de una TR, el sistema determina que estrategia usar para subsanar la caída de una TR. 
\linea \subsection*{Subsanando caída de terminal remota}
Dada una caida detectada de una terminal remota el sistema debe determinar una estretegia para subsanar la misma.
\subsubsection*{Caso de Aceptación}
Completar...

%CU 11.generando tr VIRTUAL
%Requerimiento: REQ 4
%Actor: abstracto
%Descripción: el sistema “generará” una nueva estación remota virtual, triangulado entre las tres TRs más cercanas a la TR caída.
\linea \subsection*{Generando teminal virtual}
Cuando sea necesario el sistema debe ser capaz de generar una terminal remota virtual 
%% ¿Lo que sigue va en el caso de aceptación.?
mediante la triagulacion de las tres terminales más cercanas.
\subsubsection*{Caso de Aceptación}
Completar...

%CU 12.Obteniendo INFORMACIÓN de sistema biggestsatelit
%Requerimiento: REQ 4
%Actor: abstracto
%Descripción: el sistema usará información provista por el sistema BiggestSatelit para subsanar la caída de una TR.
\linea \subsection*{Obteniendo información de sistema BiggestSatelit}
Cuando sea necesario el sistema debe poder utilizar la informacion provista por el sistema BiggestSatelit
\subsubsection*{Caso de Aceptación}
Completar...

%CU 13. detectando RECUPERACIÓN de tr
%Requerimiento: REQ 4
%Actor: TR
%Descripción: el sistema detectará la recuperación de una TR y dejará de realizar las acciones que se estaban ejecutando para subsanar la caída de una TR.
\linea \subsection*{Detectando recuperacion de terminal}
El sistema debe ser capaz de detectar la recuperación de una terminal caida y actuar en consecuencia.
\subsubsection*{Caso de Aceptación}
%% actuar en concecuencia ?
Completar...

%CU 14.informando la utilizaciÓn del sistema biggestsatelit 
%Requerimiento: REQ 5
%Actor: Ministerio de Infraestructura
%Descripción: el sistema almacenará y permitirá recuperar la información pertinente sobre el uso del sistema BiggestSatelit, cuando el ministerio de Infraestructura lo requiera.
\linea \subsection*{Informando la utilización del sistema BiggestSatelit}
El sistema deberá llebar constancia detallada del uso del sistema BiggestSatelit y se deberá informar al ministerio de Infraestructura cuando este lo requiera.
\subsubsection*{Caso de Aceptación}
Completar...

%CU 15. detectando anomalía en datos recolectados
%Requerimiento: REQ 8
%Actor: abstracto
%Descripción: el sistema detectará un dato anómalo contrastándolo con los datos históricos.
\linea \subsection*{Detectando anomalía en datos recolectados}
Ante un dato anómalo el sistema debe ser capaz de detectarlo.
\subsubsection*{Caso de Aceptación}
Completar...

%CU 16. notificando anomalía en datos 
%Requerimiento: REQ 8
%Actor: abstracto
%Descripción: el sistema deberá informar la detección de un dato anómalo a los expertos para que estos determinen si debe incorporarse este dato al modelo o no. 
\linea \subsection*{Notificando anomalía en datos}
Ante un dato anómalo detectado por el sistema, tal hecho debe ser informado a los expertos relacionados a tal fin.
\subsubsection*{Caso de Aceptación}
Completar...

%CU 17. aplicando DECISIÓN sobre anomalía
%Requerimiento: REQ 8
%Actor: experto
%Descripción: el experto tomará una decisión (aceptar o no) sobre un dato anómalo y este será registrado o descartado (en el sistema) según esta decisión.
\linea \subsection*{Aplicando Desición sobre anomalía}
Ante la respuesta de un experto sobre un dato, el mismo deberá encargarse a aplicar la misma.
\subsubsection*{Caso de Aceptación}
Completar...

%CU 18. recolectando datos de sistema eólico
%Requerimiento: REQ 9
%Actor: sistema eólico
%Descripción: el sistema incorporará e integrará los datos producidos por el sistema eólico, con los datos propios, priorizando aquellos que se consideren que tengan mayor precisión.
\linea \subsection*{Recolectando datos de sistema eólico}
El sistema se debe se capaz de integrarse con el sistema eólico para compartir información.
\subsubsection*{Caso de Aceptación}
Completar...

%CU 19. Visualizando INFORMACIÓN de monitoreo
%Requerimiento: REQ 10
%Actor: operario
%Descripción: el usuario ingresará a la herramienta de monitoreo, donde podrá visualizar el estado de las TRs, sensores y nivel de tráfico en un mapa. 
\linea \subsection*{Visualizando información de monitoreo}
Los administradores del sistema deben poder visualizar el estado de las terminales, sensores y nivel de tráfico en un mapa.
\subsubsection*{Caso de Aceptación}
%% Ver que onda la sitribución del sistema
Completar...

%CU 20. configurando NOTIFICACIÓN de alarmas 
%Requerimiento: REQ 11
%Actor: operario
%Descripción: el operario podrá cambiar la configuración de envió de notificaciones ante fallas.
\linea \subsection*{Configurando notificación de alarmas}
Los operarios del sistema deben poder cambiar la configuración de envio de notificaciones ante fallas.
\subsubsection*{Caso de Aceptación}
Completar...

%CU 21. notificando Problemas en la RED  
%Requerimiento: REQ 11
%Actor: abstracto
%Descripción: el sistema deberá informar mediante un mail la caída de una TR o demoras en determinado segmento de la red, a los usuarios determinados.
\linea \subsection*{Notificando problemas en la red}
El sistema debe informar mediante un mail la caída de una terminal o demoras en un determinado segmento.
\subsubsection*{Caso de Aceptación}
%%  a quién?
Completar...

%CU 22. calculando tráfico de red 
%Requerimiento: REQ 10 y 11
%Actor: abstracto
%Descripción: el sistema deberá obtener el nivel de tráfico en cada segmento en la red para poder determinar si se producen demoras.
\linea \subsection*{Calculando tráfico de red}
El sistema debe poder informar el nivel de tráfico en cada segmento de la red.
\subsubsection*{Caso de Aceptación}
Completar...

%CU 23.detectando demora en tráfico de red
%Requerimiento: REQ 10 y 11
%Actor: abstracto
%Descripción: el sistema deberá obtener mediante la información de tráfico en cada segmento de la red y las capacidades de tráficos que estos poseen, si se están produciendo demoras en los envíos de datos.
\linea \subsection*{Detectando demora en tráfico de red}
El sistema debe ser capaz de detectar una demora de tráfico en la red.
\subsubsection*{Caso de Aceptación}
Completar...

%CU 24.Actualizando información de la pagina web
%Requerimiento: REQ 12
%Actor: Estación Central
%Descripción: el sistema actualizará la información mostrada en el sitio Web del Ministerio de Infraestructura.
\linea \subsection*{Actualizando información de la página web}
El sistema deberá mantener actualizada la información que se muestra en el sitio web del Ministerio de Infraestructura.
\subsubsection*{Caso de Aceptación}
Completar...

%CU 25. PROVEYENDO INFORMACIÓN a cliente externo 
%Requerimiento: REQ 13
%Actor: cliente externo
%Descripción: el cliente externo solicitará un conjunto de datos que le interese y el sistema deberá proveérselos.
\linea \subsection*{Proveyendo información a cliente externo}
Ante un pedido de informacion por parte de un cliente externo, el sistema debe ser capaz de satisfacer el mismo.
\subsubsection*{Caso de Aceptación}
Completar...

%CU 26. cargando modelos de clientes externos 
%Requerimiento: REQ 13
%Actor: operario
%Descripción: el cliente proveerá un modelo propio desarrollado por él (similar a nuestro modelo de reglas), el operario se encargará de cargarlo y el sistema deberá integrarlo, para poder aplicarlo sobre los datos almacenados.
\linea \subsection*{Cargando modelos de clientes externos}
Los operarios del sistema deben poder cargar modelos de clientes externos en el sistema para que este los ejecute.
\subsubsection*{Caso de Aceptación}
Completar...

%CU 27. aplicando modelos de clientes externos
%Requerimiento: REQ 13
%Actor: Estación Central
%Descripción: el sistema deberá aplicar las reglas definidas por el modelo del cliente sobre los datos y obtener la información predictiva que está comunicada al mismo. 
\linea \subsection*{Aplicando modelos de clientes externos}
El sistema debre ser capaz de aplicar las reglas definidas por modelos de clientes externos al sistema.
\subsubsection*{Caso de Aceptación}
Completar...

%%%%%%%%%%%%%%%%%%%%%%%%%%%%%%%%%%%%%%%%%%%%%%%%%%%%%%%%%%%%%%%
%%% User Stories Agregados por cambios en req's			%%%
%%%%%%%%%%%%%%%%%%%%%%%%%%%%%%%%%%%%%%%%%%%%%%%%%%%%%%%%%%%%%%%

\linea 
\subsection*{Monitoreo provincial de Terminales}
Como organizmo provincial se desea poder monitorear las terminales del area propia de influencia.
\subsubsection*{Caso de Aceptación}

\linea 
\subsection*{Propiedad de datos}
Como organizmo provincial se desea ser propietaio %Acá aclarame qué quiere decir el enunciado con que los datos sean propios
de los datos de las terminales del área, como así también sobre la ejecución de los modelos y los resultados de estos.
\subsubsection*{Caso de Aceptación}

\linea 
\subsection*{Colaboracion entre modelos}
Se desea dar la posibilidad de interacción entre los modelos que lo requieran.
\subsubsection*{Caso de Aceptación}

\linea 
\subsection*{Modelos particionados}
Los modelos matemáticos que se encuentren particionados requieren que el sistema les permita poder interactuar para lograr un resultado final.
\subsubsection*{Caso de Aceptación}

\linea 
\subsection*{Configuracion de colaboracion intra-modelo}
El sistema debe ofrecer una forma sencilla para configurar la forma en la que los distintos sub-modelos se relacionan.
\subsubsection*{Caso de Aceptación}

\linea 
\subsection*{Varios algoritmos de evaluación}
El sistema debe ser capaz de evaluar los modelos con distintas reglas de evaluación para los mismos.
\subsubsection*{Caso de Aceptación}

\linea 
\subsection*{Subscripción a datos}
El sistema debe dar la posibilidad de que los modelos se subscriban a los datos que son de su interés sin necesidad de que le lleguen más que los que estrictamente requiere.
\subsubsection*{Caso de Aceptación}

%%% No sé si hay alguna otra Funcionalidad que se pida y que me esté olvidando... creo que no.

